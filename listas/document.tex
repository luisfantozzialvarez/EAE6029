% !TeX document-id = {1c0b4298-276a-4038-8e08-c4a1f4846da7}
% !TeX TXS-program:bibliography = txs:///biber
\documentclass[10pt,a4paper]{article}
\usepackage[T1]{fontenc}
\usepackage{graphicx}
\usepackage{mathtools}
\usepackage{amssymb}
\usepackage{amsthm}
\usepackage{thmtools}
\usepackage{xcolor}
\usepackage{nameref}
\usepackage{hyperref}
\usepackage{color}
\usepackage{float}

\usepackage[
backend=biber,
style=authoryear,
natbib=true
]{biblatex}
\addbibresource{../bibliography.bib}

\title{}
\author{\normalsize Exercícios sobre Modelos Lineares}
\date{}
\begin{document}
	\maketitle
	No que segue, sempre suponha que as variáveis aleatória relevantes estão definidas no mesmo espaço de probabilidade.
 \paragraph{Exercício 1} Seja $Z$ uma variável aleatória discreta, que toma valores no conjunto $\{z_1,\ldots, z_n\}$; e $Y$ uma variável aleatória com segundo momento finito. Defina o vetor aleatório $X  = (\mathbf{1}_{\{z_2\}}(Z),\ldots \mathbf{1}_{\{z_2\}}(Z))$. Mostre que o melhor preditor linear de $Y$ como função de um intercepto e $X$ é da forma:
 
 $$\alpha_* + \delta_*'X$$
 com
 $$ \alpha_* =  \mathbb{E}[Y|Z=z_1]$$
$$\delta_j =   \mathbb{E}[Y|Z=z_{j+1}]-\mathbb{E}[Y|Z=z_1]$$
onde definimos o objeto $\mathbb{E}[Y|Z=z_j]$ como:
$$\mathbb{E}[Y|Z=z_j] \coloneqq \frac{\mathbb{E}[Y\mathbf{1}_{\{z_j\}}(Z)]}{\mathbb{P}[\{Z = z_j\}]}\, .$$. Conclua que o melhor preditor linear coincide com $\mathbb{E}[Y|Z]$.

\paragraph{Exercício 2} Seja $h^*$ a função de esperança condicional de uma variável aleatória $Y$ com segundo momento finito dado uma variável aleatória real $X$, i.e. $h^*(X) = \mathbb{E}[Y|X]$. Compute o melhor preditor linear de $Y$ como função de um intercepto e $X$, para os casos abaixo. Ilustre graficamente a derivada de $h^*$ comparando-a com a inclinação do melhor preditor linear, nos pontos do suporte\footnote{O suporte de uma variável aleatória $X$ é definido como o menor conjunto fechado $C$ tal que $\mathbb{P}[X \in C]=1$.} de $X$. Identifique as regiões onde o melhor preditor linear mais ``erra'' a inclinação de $h^*$.

\begin{itemize}
	\item[a] $h^*(x) = x^2 $ e $X\sim U[0,1]$.
	\item[b] $h^*(x) = \min\{x^3,100\}$ e $X \sim \operatorname{Exponecial}(1)$.
	\item[c] $h^*(x) = x^2$ e $X \sim \operatorname{\chi}^2(1)$.
\end{itemize}

 \paragraph{Exercício 3} Seja $A$ uma matriz simétrica positiva semidefinida:
 \begin{itemize}
 	\item[a] Mostre que os autovalores de $A$ sempre são não negativos. \text{Dica: } Se $v\neq 0$ é um autovetor de um autovalor $\lambda$, $Av=\lambda v$.
 	\item[b] Mostre que o menor autovalor de $A$ é zero se, e somente se, existe $v \neq 0$, $v'Av = 0$. \text{Dica: } para suficiência, utilize a decomposição espectral de uma matriz simétrica.
 \end{itemize} 
 
 \paragraph{Exercício 4} Seja $(\Omega, \Sigma,\mathbb{P})$ um espaço de probabilidade, em que cada elemento $\omega \in \Omega$ corresponde a uma firma num setor de interesse, e $\mathbb{P}[A]$ denota a proporção de firmas do ``tipo'' A . Considere o exemplo, visto em aula, em que a função de produção no setor é $h(z,l) = zl^{\alpha}$, $\alpha < 1$. Suponha que, no momento da decisão de produção, uma firma $\omega \in \Omega$ na população somente observe um sinal $\xi(\omega)$ de sua produtividade. A distribuição do sinal $\xi$ na população é denotada por $\mathbb{P}_\xi$. As firmas então escolhem $l$ otimizando:
 
 $$l(\omega) \in \operatorname{argmax}_{l \geq 0} \xi(\omega) l^{\alpha}- w l\, ,$$
onde $w$ denota o salário (em unidades do bem produzido pelo setor), aqui tratado como constante fixa.\footnote{Em outras palavras, a incerteza capturada pelas variáveis aleatórias só se deve à amostragem no sentido clássico, e não à incerteza econômica.} Após a escolha de demanda por trabalho, a produtividade de cada firma é revelada como $z(\omega) = \xi(\omega)\epsilon(\omega)$, onde o ``choque'' ou ``surpresa'' na produtividade  não é antecipável com base no sinal, i.e. $\xi$ é independente de $\epsilon$  ($\mathbb{P}_{(\xi,\epsilon)}  = \mathbb{P}_{\xi}\otimes \mathbb{P}_{\epsilon}$). A produção final das firmas é dada por $Y(\omega) = z(\omega) l(\omega)$.

\begin{itemize}
	\item[a] Encontre o melhor preditor linear da variável aleatória $\log(Y)$ como função de um intercepto e $\log(l)$. O coeficiente associado a $\log(l)$ coincide com $\alpha$? Por quê?
	\item[b] Suponha que, caso $Y(\omega) - wl(\omega)<0$, a firma $\omega$ não opera, e, nesse caso, a demanda por trabalho e quantidade produzida observadas são zero. Calcule o melhor preditor linear nesse caso. Ele corresponde ao obtido anteriormente? Por quê?
\end{itemize}


 

\end{document}
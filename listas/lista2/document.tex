% !TeX document-id = {1c0b4298-276a-4038-8e08-c4a1f4846da7}
% !TeX TXS-program:bibliography = txs:///biber
\documentclass[10pt,a4paper]{article}
\usepackage[T1]{fontenc}
\usepackage{graphicx}
\usepackage{mathtools}
\usepackage{amssymb}
\usepackage{amsthm}
\usepackage{thmtools}
\usepackage{xcolor}
\usepackage{nameref}
\usepackage{hyperref}
\usepackage{color}
\usepackage{float}

\usepackage[
backend=biber,
style=authoryear,
natbib=true
]{biblatex}

\usepackage{geometry}
\addbibresource{../bibliography.bib}

\title{}
\author{\normalsize Exercícios sobre Estimação por MQO}
\date{}
\begin{document}
	\maketitle
	No que segue, sempre suponha que as variáveis aleatória relevantes estão definidas no mesmo espaço de probabilidade.
 \paragraph{Exercício 1} Seja $S$ um vetor aleatório, e $X$ um conjunto de variáveis aleatórias. Mostre a lei da variância total, i.e:
 
 $$\mathbb{V}[S] = \mathbb{E}[\mathbb{V}[S|X]] + \mathbb{V}[\mathbb{E}[S|X]]\, .$$

 \paragraph{Exercício 2} Sejam $A$ e $B$ duas matrizes simétricas positiva definidas de dimensão $n$. Mostre que, se $A-B$ é positiva semidefinida, então $B^{-1} - A^{-1}$ é positiva semidefinida.
 
 \paragraph{Exercício 3} Considere o seguinte modelo linear para uma amostra de $n$ observações:
 
 $$\boldsymbol{y} = \boldsymbol{X}_1 \beta_1 + \boldsymbol{X}_2 \beta_2 + \boldsymbol{\epsilon}\,,$$
 onde $\mathbb{E}[\boldsymbol{\epsilon}|\boldsymbol{X}_1,\boldsymbol{X}_2]=0$, $\mathbb{V}[\boldsymbol{\epsilon}|\boldsymbol{X}]=\sigma^2\mathbb{I}_{n\times n}$, $\left[\boldsymbol{X}_1 \, \, \boldsymbol{X}_2\right]$ tem posto cheio, e $\boldsymbol{X}_1 = \left[ \boldsymbol{1}_{n\times 1} \, \, \boldsymbol{Z}\right]$, onde  $\boldsymbol{1}_{n\times 1}  $ é um vetor de uns e $\boldsymbol{Z}$ é \emph{independente} de $\boldsymbol{X}_2$.
 
 \begin{itemize}
 	\item[a] Mostre que o estimador de MQO de $\boldsymbol{y}$ em $\boldsymbol{X}_1$, denotado por $\tilde{b}_1$, é não viciado (condicionalmente a $\boldsymbol{X}_1$) para $\beta_1$.
 	\item[b] Mostre que a variância condicional de $\tilde{b}_1$ se escreve como:
 	
 	$$\mathbb{V}[\tilde{b}_1|\boldsymbol{X}_1] = 	(\boldsymbol{X}_1'\boldsymbol{X}_1)^{-1}\boldsymbol{X}_1'\mathbb{V}[\boldsymbol{X}_2\beta_2]\boldsymbol{X}_1'(\boldsymbol{X}_1\boldsymbol{X}_1)^{-1} + \sigma^2 (\boldsymbol{X}_1'\boldsymbol{X}_1)^{-1}$$
 	
 	\item[c] Mostre que a variância do estimador de MQO de $\beta_1$ da regressão que inclui $\boldsymbol{X}_1$ e $\boldsymbol{X}_2$, denotado por $\hat{b}_1$, é não viciado para $\beta_1$ (condicionalmente a $\boldsymbol{X}_1$ e $\boldsymbol{X}_2$) e que sua variância condicional é dada por:
 	
 	$$\mathbb{V}[\hat{b}_1|\boldsymbol{X}_1,\boldsymbol{X}_2] = \sigma^2\left(\boldsymbol{X}_1'M_2 \boldsymbol{X}_2\right)^{-1}\, , $$
 	onde $M_2$ é a residualizadora de $M_2$.
 	
 	\item[d] Mostre que $\left(\boldsymbol{X}_1'M_2 \boldsymbol{X}_2\right)^{-1} - (\boldsymbol{X}_1'\boldsymbol{X}_1)^{-1}$ é positiva semidefinida.
 	
 	\item[e] Mostre que $(\boldsymbol{X}_1'\boldsymbol{X}_1)^{-1}\boldsymbol{X}_1'\mathbb{V}[\boldsymbol{X}_2\beta_2]\boldsymbol{X}_1'(\boldsymbol{X}_1\boldsymbol{X}_1)^{-1} $ é positiva semidefinida.
 	
 	\item[f] Usando a  lei da variância total e os dois resultados anteriores, mostre que a diferença entre as variâncias \emph{incondicionais}de $\tilde{b}_1$ e $\hat{b}_1$ se fatoram em dois termos, $A$ e $B$, em que $A$ é a esperança de uma matriz aleatória positiva semidefinida e $B$ a esparança de uma matriz negativa semidefinida. Proveja uma intuição para esse resultado. Se o interesse reside em $\beta_1$, quando fará sentido incluir $\boldsymbol{X}_2$ na regressão?
 	\item[g] Mostre que se $\beta_2 = 0$, então  $\mathbb{V}[\hat{\beta}_1] - \mathbb{V}[\tilde{\beta}_1] $ é positiva semidefinida. Esse resultado contradiz o teorema de Gauss-Markov? Por quê?
 \end{itemize}
 
  \paragraph{Exercício 4} Suponha que você esteja interessado em estudar o efeito de infraestrutura escolar no desempenho escolar médio dos aluno. Você possui acesso a uma amostra aleatória de municípios, para os quais você observa o desempenho escolar médio dos alunos de $C$ escolas, $c=1,\ldots, C$, em cada um dos $M$ munícipios,  $m=1,\ldots, M$, amostrados ($Y_{c,m}$), e uma medida de conectividade da escola à Internet, em \textit{megabytes por segundo} (mbps) ($X_{c,m}$). Você então postula o seguinte modelo causal linear para $Y_{c,m}$
  
  $$Y_{c,m} = \gamma X_{c,m} + \epsilon_{c,m}\,,\quad c=1, \ldots,  C, \, \, m=1,\ldots, M\, ,$$
 onde $\gamma$ é o efeito causal de se aumentar a conectivadade da escola em 1mbps sobre o desempenho médio dos alunos, e $\epsilon_{c,m}$ são as demais causas não observadas do desempenho escolar.
 
 \begin{itemize}
 	\item[a] Sob qual condição a inclinação do melhor preditor linear de $Y_{c,m}$ num intercepto e $X_{c,m}$ identifica $\gamma$? Você acredita que essa condição seria satisfeita na prática? Por quê?
 \end{itemize}
 \vspace{0.5 em}
 
 Em virtude da dificuldade de se acreditar na hipótese anteriormente postulada, um colega seu sugere que você explicitamente acomode a presença de causas não observadas no nível municipal que afetam desempenho municipal, considerando a decomposição:
 
 $$\epsilon_{c,m} = \psi_m + \nu_{c,m}\, , $$
 onde $\psi_m$ é um conjunto de causas comuns ao município, e $\nu_{c,m}$ são causas idiossincráticas às escolas. Seu colega lhe diz que é possível levar em conta  (``controlar'') as causas $\psi_m$ na estimação, mesmo que estas não sejam observadas, considerando uma regressão de $Y_{c,m}$ em $X_{c,m}$ e um conjunto de $M$ \textit{dummies} municipais, i.e. $D_{c,m}(l) =\mathbf{1}_{\{l\}}(m)$, $l=1\,\ldots, m$.
 
 \begin{itemize}
 	\item[b] Usando o teorema de Frisch-Waugh-Lovell, mostre que o estimador do coeficiente de $X_{c,m}$ sugerido pelo seu colega é algebricamente idêntico a se estimar $\gamma$ a partir de uma regressão (sem intercepto) de $Y_{c,m} - \bar{Y}_m$ em $X_{c,m} - \bar{X}_m$, onde $\bar{Z}_m = \frac{\sum_{i=1}^{C}}{C} Z_{i.m}$. Em outras palavras, o estimador sugeriod pelo seu colega é equivalente ao estimador de MQO do modelo transformado.
 	
 	$$Y_{c,m} - \bar{Y}_m = \gamma (X_{c,m}-\bar{X}_m) + \epsilon_{c,m}-\bar{\epsilon}_m\, .$$
 	
 	\item[c] Proveja condições suficientes para que o estimador de MQO sugerido seja consistente para o parâmetro de interesse, num regime assintótico em que $M \to \infty$. Qual a interpretação dessas condições?
 	
 	\item[d] Derive a variância assintótica do estimador, sob as hipóteses do item anterior, permitindo que haja correlação arbitrária entre os $\epsilon_{c,m}$ intramunicipais. Sugira um estimador consistente para a variância assintótica.
 \end{itemize}
 
 
 \paragraph{Exercício 5} O arquivo \texttt{cps\_union\_data.csv} conté uma amostra \textbf{12.834 indivíduos na força de trabalho norte-americana}, extraída do suplemento anual de 2019 da \textit{Pesquisa de População Atual dos EUA (Current Population Survey)}. Seu objetivo é \textbf{estimar o efeito causal da filiação/cobertura sindical} (variável \texttt{union}) sobre os \textbf{rendimentos semanais} (variável \texttt{earnings}). O conjunto de dados contém diversas outras variáveis que podem ser utilizadas como controles (verifique o dicionário do conjunto de dados, disponível em \texttt{dictionary.xlsx}).
 
 \begin{enumerate}
 	
 	\item Como ponto de partida, compare a \textbf{média dos rendimentos} entre os indivíduos com cobertura sindical (\texttt{union == 1}) e os indivíduos sem essa cobertura (\texttt{union == 0}). Qual é a diferença estimada? Usando um teste $t$ para comparação de médias entre duas populações, teste a hipótese nula de que a remuneração média na população filiada é igual à remuneração média na população não filiada, contra a alternativa bilateral. Você rejeita a hipótese nula a 5\% de significância? E a 1\%?  diferença é estatisticamente significativa? Você acredita que essa diferença é uma estimativa crível do impacto causal da cobertura sindical? Por quê? \textit{Dica:} utilize o comando \texttt{t.test} do pacote básico do \texttt{R}.
 	
 	\item Mostre analiticamente que a diferença de médias estimada anteriormente pode ser obtida através de uma regressão de \texttt{earnings} em um intercepto e \texttt{union}, e que a estatística $t$ do coeficiente associado a \texttt{union} baseada em erros padrão robustos à heterocedasticidade é idêntica, a não ser por correção de graus de liberdade, à estatística $t$ utilizada no teste anterior. 
 	
 	\item Para melhorar e/ou avaliar a credibilidade dos seus resultados anteriores, você decide considerar um modelo linear causal da forma:
 	
 	\begin{equation}
 		\label{regression_model}
 		\text{earnings}_i = \beta_0 + \beta_1 \text{union}_i + \gamma'Z_i + \epsilon_{it}
 	\end{equation}
 	
 	onde $Z_i$ representa um conjunto de variáveis de controle.
 	
 	\begin{enumerate}
 		\item[a)] Especifique o modelo linear da Equação \eqref{regression_model} apresentando um conjunto de covariáveis $Z$ a serem incluídas como controles. Justifique a escolha dessas variáveis. Qual é a interpretação de $\beta_1$ no seu modelo?
 		
 		\item[b)] Estime o modelo especificado. Qual é a sua estimativa de $\beta_1$? Usando erros padrão robustos à heterocedasticidade, ela é estatisticamente significativa? Comente brevemente os seus resultados.
 		

 			\item[c)]Usando o teorema de Frisch-Waugh-Lovell, mostre que o estimador de MQO para $\beta_1$ tem a seguinte forma:
 		
 		\begin{equation}
 			\hat{\beta}_1 = \sum_{i=1}^N  \text{union}_i \cdot \omega_i  \cdot \text{earnings}_i -  \sum_{i=1}^N (1-  \text{union}_i )\cdot \omega_i  \cdot \text{earnings}_i 
 		\end{equation}
 		
 		onde os pesos $\omega_i$ são definidos como:
 		
 		\begin{equation}
 			\omega_i  = \frac{\hat{\xi}_i (2 \cdot \text{union}_i-1)}{\text{SSR}_{\text{union},Z}}
 		\end{equation}
 		
 		com $\hat{\xi}_i$ sendo o resíduo da observação $i$ de uma regressão linear de $\text{union}_i$ sobre $Z$ (incluindo intercepto); e $\text{SSR}_{\text{union},Z}$ é a soma dos quadrados dos resíduos dessa regressão auxiliar.
 		
 		\item[d)] Calcule os pesos para sua especificação usando a fórmula acima. Apresente estatísticas descritivas da distribuição dos pesos nos grupos de controle e tratamento. 
 		
 		\begin{itemize}
 			\item Os pesos somam 1 no grupo de controle?
 			\item E no grupo de tratamento?
 			\item Há valores negativos?
 			\item E outliers?
 		\end{itemize}
 		
 		Como esses pesos diferem com relação ao estimador de diferença de médias para o tratamento? Proveja uma intuição para essa diferença.
 		
 	\end{enumerate}
 	
 \end{enumerate}

 \paragraph{Exercício 6} O objetivo deste exercício consiste em entender as propriedades do estimador da variância robusta a \textit{cluster}. Para isso, usaremos os dados do resultado da aplicação de um teste lógico em alunos de um conjunto de escolas primárias, disponível em \texttt{dados.csv}.
 \begin{itemize}
 	\item[a] Construa uma função em \texttt{R} que:
 	\begin{enumerate}
 		\item Sorteie 500 escolas, $c=1,\ldots, 500$, \textbf{com reposição}, do conjunto de dados. Note que haverá escolas repetidas, mas elas serão tratadas como distintas para os fins da simulação (de modo a refletir amostragem aleatória de uma população de escolas).
 		\item Para cada uma das $c=1,\ldots, 500$ escolas sorteadas, gere um tratamento fictício no nível escolar, $X_c \sim \operatorname{Uniforme}[0,1]$, de forma independente entre as escolas e dos dados.
 		\item Com base na base de dados construída, estime por MQO o modelo linear:
 		
 		$$\text{logico}_{i,c} = \alpha + \beta X_{c} + \gamma \text{sexo}_{i,c} + \phi \text{idade}_{i,c} + \epsilon_{i,c}\, ,$$
 		\item Guarde as estatísticas $t$ do teste da hipótese nula de que $\beta =0$ baseados em erros padrão robustos à heterocedasticidade e a \textit{cluster} no nível da escola.
 	\end{enumerate}
 	\item[b] Aplique a função 1000 vezes, e calcule a proporção de casos em que cada um dos dois tipos de testes $t$, a 5\% de significância, rejeita a hipótese nula. Se o estimador da variância estiver ``correto'', qual a proporção esperada de casos em que se rejeita a nula?  Por quê? O que você obteve na prática? Por quê?
 	
 	\item[c] Agora adapte a função anterior, permitindo que o tratamento, agora denotado por $\tilde{X}_{i,c}\sim \operatorname{Uniforme}[0,1]$, varie no nível individual de forma independente. Aplique a nova função mil vezes. Reporte a proporção de casos em que se rejeitou a hipótese nula. O que ocorreu? Por quê?
 	\item[d] Adapte mais uma vez a função, agora permitindo que o tratamento exerça um efeito heterogêneo entre escolas. Especificamente, você gerará $\widetilde{\texttt{logico}}_{i,c} = \texttt{logico}_{i,c}  + \tau_c \tilde{X}_{i,c}$, onde $\tau_c = 1$ com probabilidade $1/2$ e -1 com probabilidade $1/2$, e independente das demais variáveis. Nós continuaremos rodando a regressão linear de $\tilde{\texttt{logico}}_{i,c}$ em $\tilde{X}_{i,c}$ e controles. Aplique a função nova 1000 vezes. Qual a proporção de casos rejeitados agora? Por quê? 
 \end{itemize}
\end{document}
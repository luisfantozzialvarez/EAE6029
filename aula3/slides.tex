% !TeX document-id = {0be8c18c-9430-4e9a-bdd9-12beadebfebc}
% !TeX TXS-program:bibliography = txs:///biber
\documentclass[11pt]{beamer}

\usepackage[brazilian]{babel}

\uselanguage{portuguese}
\languagepath{portuguese}
\deftranslation[to=portuguese]{Theorem}{Teorema}
\deftranslation[to=portuguese]{theorem}{teorema}
\deftranslation[to=portuguese]{Example}{Exemplo}
\deftranslation[to=portuguese]{example}{exemplo}
\deftranslation[to=portuguese]{Lemma}{Lema}
\deftranslation[to=portuguese]{lemma}{Lema}
\deftranslation[to=portuguese]{Corollary}{Corolário}
\deftranslation[to=portuguese]{corollary}{corolário}
%\deftranslation[to=portuguese]{and}{e}
\newtheorem{proposition}{Proposição}
\newtheorem{assumption}{Hipótese}

\usepackage[utf8]{inputenc}
\usepackage[T1]{fontenc}
\usepackage{lmodern}
\usepackage{amsmath}
\usepackage{amssymb}
\usepackage{mathtools}
\usepackage{color}
\usepackage{pgfplots}
\usepackage{tikz}
\usepackage{subcaption}
%\usepackage{appendixnumberbeamer}

\newenvironment{transitionframe}{
	\setbeamercolor{background canvas}{bg=yellow}
	\begin{frame}}{
	\end{frame}
}
\usetheme{default}
\usefonttheme{structuresmallcapsserif}

%% I use a beige off white for my background
\definecolor{MyBackground}{RGB}{255,253,218}
\useinnertheme[shadow]{rounded}
\setbeamercolor{block title}{bg=MyBackground}
\setbeamercolor{block body}{bg=MyBackground}
\setbeamercolor{example title}{bg=MyBackground}
\setbeamercolor{example body}{bg=MyBackground}


\newcommand{\blue}[1]{\textcolor{blue}{#1}}
\newcommand{\red}[1]{\textcolor{red}{#1}}
\newcommand{\purple}[1]{\textcolor{purple}{#1}}
\newcommand{\gray}[1]{\textcolor{gray}{#1}}
\setbeamertemplate{navigation symbols}{}
%\setbeamertemplate{page number in head/foot}[appendixframenumber]

%\usepackage{graphics}
\usepackage{graphicx}

\definecolor{blue_emph}{RGB}{0,114,178}
\definecolor{red}{RGB}{213,94,0}
\definecolor{yellow}{RGB}{240,228,66}
\definecolor{green}{RGB}{0,158,115}
\definecolor{purple}{RGB}{204,121,167}
\definecolor{orange}{RGB}{230,159,0}
\definecolor{lightblue}{RGB}{86,180,233}

%\setbeamercolor{frametitle}{fg=blue}
%\setbeamercolor{title}{fg=blue}
\setbeamertemplate{footline}[frame number]
\setbeamertemplate{navigation symbols}{} 
\setbeamertemplate{itemize items}{-}
%\setbeamercolor{itemize item}{fg=blue}
%\setbeamercolor{itemize subitem}{fg=blue}
\setbeamertemplate{enumerate items}[default]
%\setbeamercolor{enumerate subitem}{fg=blue}
\setbeamercolor{button}{bg=MyBackground,fg=blue}
\usefonttheme{structuresmallcapsserif}

%\setbeamercolor{section in toc}{fg=blue}
%\setbeamercolor{subsection in toc}{fg=red}
\setbeamersize{text margin left=1em,text margin right=1em} 


\usepackage{appendixnumberbeamer}
\usepackage{pdfpages}
\usepackage[
backend=biber,
uniquename=false,
uniquelist=false,
style=authoryear,
natbib=true
]{biblatex}
\addbibresource{../bibliography.bib}

\newenvironment{wideitemize}{\itemize\addtolength{\itemsep}{10pt}}{\enditemize}
\newenvironment{wideenumerate}{\enumerate\addtolength{\itemsep}{10pt}}{\endenumerate}
\newenvironment{halfwideitemize}{\itemize\addtolength{\itemsep}{0.5em}}{\enditemize}
\newenvironment{halfwideenumerate}{\enumerate\addtolength{\itemsep}{0.5em}}{\endenumerate}


\author{Luis A. F. Alvarez}
\title{Econometria I}
\subtitle{Variáveis instrumentais}
%\logo{}
%\institute{}
\date{\today}
%\subject{}
%\setbeamercovered{transparent}

\def\signed #1{{\leavevmode\unskip\nobreak\hfil\penalty50\hskip2em
		\hbox{}\nobreak\hfil(#1)%
		\parfillskip=0pt \finalhyphendemerits=0 \endgraf}}

\newsavebox\mybox
\newenvironment{aquote}[1]
{\savebox\mybox{#1}\begin{quote}}
	{\signed{\usebox\mybox}\end{quote}}

\begin{document}

\begin{frame}[plain]
	\maketitle
\end{frame}
\begin{frame}{Instrumento}
	\begin{itemize}
		\item Considere um modelo linear {\color{blue}causal} da forma:
		
		$$Y = X\beta + U\, ,$$
		onde $X$ é uma causa observada escalar, $U$ são causas não observadas, e $\beta$ é o efeito causal de $X$.
		\item Considere uma situação em que não é razoável supor que $\operatorname{cov}(X,U) = 0$.
		\begin{itemize}
			\item Nesse caso, a inclinação de $X$ no melhor preditor linear de $Y$ em $X$ e um intercepto não identificará $\beta$, de modo que o estimador de MQO de $Y$ em $X$ e um intercepto não estimará consistentemente o efeito causal de interesse.  
		\end{itemize}
		\item Suponha que observemos uma variável $Z$ que satisfaz:
		\begin{itemize}
			\item (Relevância) $\operatorname{cov}(Z,X)\neq0$.
			\item (Exogeneidade ou exclusão)  $\operatorname{cov}(Z,U)=0$
		\end{itemize}
		\item À variável $Z$ damos o nome de {\color{green}instrumento}:
		\begin{itemize}
			\item Trata-se de variável que exibe associação, na população, com $X$ (relevância), e cuja \textbf{única associação com $Y$ se dá através de $X$} (exogeneidade ou exclusão).
		\end{itemize}
\end{itemize}
	\end{frame}
\begin{frame}{Estimando de Wald e Identificação sob variável instrumental}
\begin{itemize}
	\item Defina  \textbf{o estimando de Wald} como o \textbf{parâmetro}:
	
	$$\gamma_{\text{Wald}}  \coloneqq \frac{\operatorname{cov}(Z,Y)}{\operatorname{cov}(Z,X)}\,,$$
	\item Sob as hipóteses de relevância e exogeneidade do instrumento, estimando identifica $\beta$, i.e.:
	
	$$\gamma_{\text{Wald}} = \beta\, .$$
\end{itemize}
\end{frame}
\begin{frame}{Estimação e Inferência}
	\begin{itemize}
		\item Dada uma amostra aleatória $(Y_i,X_i,Z_i)\sim (Y,X,Z)$, resultado anterior sugere que estimemos o efeito causal como:
		
		$$\hat{\gamma}_{\text{Wald}} = \frac{\widehat{\operatorname{cov}(Z,Y)}}{\widehat{\operatorname{cov}(Z,X)}} = \frac{\hat{b}_{Y,Z}}{\hat{b}_{X,Z}}\, ,$$
		onde $\hat{b}_{S,Z}$ é o estimador de MQO para o coeficiente de $Z$ numa regressão de $S$ num intercepto e $Z$.
		\item Sob condições de regularidade da aula anterior, vimos que, com $n \to \infty$ $$\hat{b}_{Y,Z} \overset{p}{\to} \frac{\operatorname{cov}(Y,Z)}{\mathbb{V}(Z)}$$
		
		$$\hat{b}_{X,Z} \overset{p}{\to} \frac{\operatorname{cov}(X,Z)}{\mathbb{V}(Z)}$$
		de onde segue, por aplicação do teorema do mapa contínuo que, sob as condições de relevância e exogeneidade:
		\vspace{-0.2em}
		$$\hat{\gamma}_{\text{Wald}}  \overset{p}{\to} \gamma_{\text{Wald}} = \beta\, .$$
		
	\end{itemize}
\end{frame}
\begin{frame}{Testando relevância}
	\begin{itemize}
		\item Note que, como estimador de $\hat{b}_{X,Z}$ é consistente para $\frac{\operatorname{cov}(X,Z)}{\mathbb{V}(Z)}$, podemos utilizar esse estimador para realizar um teste da nula de que que o {instrumento} \textbf{não} é relevante.
		\item Por outro lado, em nosso ambiente com uma variável instrumental e um modelo causal linear, a hipótese de {exclusão} é \textbf{intestável}.
		\begin{itemize}
			\item Devemos suplementar a análise empírica com uma argumentação de porquê o único mecanismo através do qual variações em $Z$ produzem variações em $Y$ é através de $X$.
		\end{itemize}
	\end{itemize}
\end{frame}

\begin{frame}{Caso geral}
\begin{itemize}
	\item Vamos agora considerar um modelo causal geral da forma:
$$Y = X'\beta + U$$
onde $X$ é um vetor de $k$ causas observadas.
\item Vamos supor a existência de um vetor $Z$  de $l$ instrumentos que satisfaz as hipóteses:
\begin{assumption}[H1-Relevância]
	$\mathbb{E}[ZX']$ tem posto $k$.
\end{assumption}

\begin{assumption}[H2-Exogeneidade]
	$\mathbb{E}[ZU]=0$
\end{assumption}
\end{itemize}

\begin{itemize}
	\item Note que a hipótese de exogeneidade é equivalente a $\operatorname{cov}(Z,U)=0$ quando $X$ inclui um intercepto, pois nesse caso podemos supor que $U$ tem média zero ``absorvendo sua média'' ao intercepto.
	\begin{itemize}
		\item Note, no entanto, que isso muda a interpretação do intercepto, que supomos não ser de interesse direto no caso.
	\end{itemize}
\end{itemize}

\end{frame}

\begin{frame}{Variáveis Endógenas e Exógenas}
\begin{itemize}
	\item A condição de relevância implica que necessitamos de $l\geq k$ variáveis que exibam suficiente variação com as causas $X$, e que não exibam associação com as causas $U$.
	\begin{itemize}
		\item Se uma entrada $X_j$ é {\color{blue}exógena} por hipótese, i.e. $\mathbb{E}[X_j U] = 0$, podemos incluí-la entre os instrumentos $Z$.
		\item Por outro lado, para cada variável $X_s$  {\color{blue}endógena}, i.e. tal que potencialmente $\mathbb{E}[X_s U] \neq 0$, precisamos de pelo menos um instrumento que exiba associação com as causas observadas, mas não com as causas não observadas.
	\end{itemize}
\end{itemize}
\end{frame}

\begin{frame}{Identificação dos efeitos no modelo linear geral}
\begin{itemize}
	\item Sob as hipóteses H1 e H2, {\color{red}para qualquer matriz $A$ $k\times l$} de posto cheio, temos que:
	$$ \gamma_A := (\mathbb{E}[AZX'])^{-1}\mathbb{E}[AZY] = \beta$$
	\item Note que, o resultado anterior sugere que, para uma amostra aleatória $(Y_i,X_i, Z_i)\sim (Y,X,Z)$ e uma dada matriz $A$, podemos estimar $\beta$ por:
	
	$$\hat{\gamma}_A = \left(\sum_{i=1}^n \hat{A} Z_i X_i'\right)^{-1}\sum_{i=1}^n \hat{A} Z_i Y_i\, = (\hat{A} \boldsymbol{Z}'\boldsymbol{X})^{-1}\hat{A} \boldsymbol{Z}' \boldsymbol{y}\, ,$$
	onde $\hat{A}$ é um estimador da matriz $A$ e:
	$$\boldsymbol{Z} = \begin{bmatrix}
		Z_1' \\
		Z_2' \\
		\vdots \\
		Z_n'
	\end{bmatrix}\, , \quad \boldsymbol{X} = \begin{bmatrix}
	X_1' \\
	X_2' \\
	\vdots \\
	X_n'
	\end{bmatrix}, \quad \boldsymbol{y} =\begin{bmatrix}
	Y_1\\
	Y_2\\
	\vdots \\
	Y_n
	\end{bmatrix}$$
\end{itemize}

\end{frame}

\begin{frame}{Mínimos Quadrados em Dois Estágios}
\begin{itemize}
	\item Note que, em termos de \textbf{identificação}, temos alguma liberdade na escolha de $\hat{A}$.
	\item Em termos \textbf{inferenciais}, uma escolha natural é tomar a combinação linear dos instrumentos $\hat{A}$ que "maximiza" o poder explicativo de $\boldsymbol{Z}$ sobre $\boldsymbol{X}$. Isto é, tomamos:
	$$\hat{A}' = (\boldsymbol{Z}'\boldsymbol{Z})^{-1}(\boldsymbol{Z}'\boldsymbol{X}) $$
	\item Essa escolha produz o estimador de {\color{blue}MQO em dois estágios}, que denotamos por $\hat{b}_{\text{2SLS}}$:
	\begin{equation*}
		\hat{b}_{\text{2SLS}} = (\boldsymbol{X}'\boldsymbol{Z}(\boldsymbol{Z}'\boldsymbol{Z})^{-1}\boldsymbol{Z}'\boldsymbol{X})^{-1} \boldsymbol{X}'\boldsymbol{Z}(\boldsymbol{Z}'\boldsymbol{Z})^{-1} \boldsymbol{Z}'\boldsymbol{y} = \left(\boldsymbol{X}'P_{\boldsymbol{Z}}\boldsymbol{X}\right)^{-1}\boldsymbol{X}'P_{\boldsymbol{Z}}\boldsymbol{y} 
	\end{equation*}
	onde $P_{\boldsymbol{Z}}$ é a matriz de projeção de $\boldsymbol{Z}$.
	\item Estimador pode ser obtido em dois estágios
	\begin{itemize}
		\item Primeiro, regredimos cada uma das $v=1\,2\dots,k$ variáveis em$\boldsymbol{X}$, $X_{i,v}$, em $Z_i$, e guardamos os valores preditos $\hat{X}_{i,v} = \hat{c}_{v}'Z_i$.
		\begin{itemize}
			\item Note que $\hat{X}_{i,v} = X_{i,v} $ se $X_{v}$ é variável inclusa em $Z$. 
		\end{itemize}
		\item Usando os valores preditos, regredimos $Y_i$ nos $\hat{X}_{i,v}$,$v=1\,l\dots,k$.  O estimador de $\hat{b}_{\text{2SLS}}$ será o resultado dessa regressão.
	\end{itemize}
\end{itemize}
\end{frame}

\begin{frame}{Propriedades Assintóticas do 2SLS}
	Para a análise do estimador de MQO em dois estágios requeremos:
	\begin{assumption}[H3-Posto]
		$\mathbb{E}[ZZ']$ tem posto cheio.
	\end{assumption}
	\begin{proposition}
		Sob as hipóteses H1-H3, amostragem aleatória $(Y_i,X_i,Z_i) \overset{iid}{\sim}(Y,X,Z)$ e segundos momentos finitos de $(Y,X,Z)$, $\hat{b}_{2SLS}\overset{p}{\to}\beta$.
	\end{proposition}
		\begin{proposition}
		Sob as hipóteses H1-H3, amostragem aleatória $(Y_i,X_i,Z_i) \overset{iid}{\sim}(Y,X,Z)$ e \textbf{quartos} momentos finitos de $(X,Y,Z)$, 
		\vspace{-0.4em}
		$$\sqrt{n}(\hat{b}_{\text{2SLS}}-\beta) \overset{d}{\to}\mathcal{N}(0,BMB)\, ,$$
		\vspace{-2.4em}
	\begin{align*}
		\footnotesize B=\left(\mathbb{E}[XZ']\mathbb{E}[ZZ']^{-1}\mathbb{E}[X'Z]\right)^{-1}\, , \\ M = \mathbb{E}[XZ']\mathbb{E}[ZZ']^{-1}\mathbb{E}[U^2 ZZ']\mathbb{E}[ZZ']^{-1}\mathbb{E}[ZX']
	\end{align*}
	\end{proposition}
\end{frame}

\begin{frame}{Eficiência do MQO em dois estágios}
A escolha de pesos $A$ implícita ao estimador de MQO em dois estágios possui propriedades de eficiência assintótica, na classe de estimadores baseados em combinações lineares dos instrumentos, se uma versão da hipótese de homocedasticidade valer.

\begin{proposition}
	Sob as hipóteses H1-H3, amostragem aleatória $(Y_i,X_i,Z_i) \overset{iid}{\sim}(Y,X,Z)$, quartos momentos finitos de $(X,Y,Z)$, e a condição:
	
	$$\mathbb{E}[U^2|Z] = \sigma^2\, .$$ 

	Então o estimador de 2SLS exibe a menor variância assintótica, relativamente a qualquer outro estimador $\hat{\gamma}_C$ baseado numa combinação linear $\hat{C}\boldsymbol{Z}$ dos instrumentos, onde $\hat{C} \overset{p}{\to} C$. Em outras palavras:
	
	$$\operatorname{Avar}(\sqrt{n}(\hat{\gamma}_{C}-\beta)) - \operatorname{Avar}(\sqrt{n}(\hat{b}_{\text{2SLS}}-\beta)) \text{ é positiva semidefinida}$$
\end{proposition}
\end{frame}

\begin{frame}{Teste de relevância}
	\begin{itemize}
		\item Assim como no estimador de Wald, é possível usar o primeiro estágio para testar a hipótese de relevância de que $\mathbb{E}[ZX']$ tem posto $k$.
		\begin{itemize}
			\item Se só há somente uma variável endógena em $X$, só há efetivamente um primeiro estágio a ser rodado (por quê?), e podemos testar relevância realizando um teste de Wald de que os coeficientes de primeiro estágio associados aos instrumentos excluídos (isto é, que não constam entre as variáveis exógenas em $X$) são todos iguais a zero.
			\item Se há mais de uma endógena, temos de fazer um teste de Wald multi-equação, que pode ser construído a partir do ``empilhamento'' dos diferentes primeiros estágios (veja o livro do Wooldridge para mais discussão).
		\end{itemize}
	\end{itemize}
\end{frame}
\begin{frame}{Teste de endogeneidade}
\begin{itemize}
	\item Se $Z \neq X$, é possível construir um teste da hipótese nula:
	$$\operatorname{cov}(X,U) = 0$$
	contra a alternativa de que $\operatorname{cov}(X,U)\neq 0$.
	\item Ideia é basear teste em estatística de Wald que meça a  ``magnitude'' de:
	
	$$\frac{1}{n}\sum_{i=1}^n X_i (Y_i - X_i'\hat{b}_{2SLS})$$
	\vspace{-1em}
	\begin{itemize}
		\item Observe que, se $Z = X$, $\hat{b}_{2SLS}=\hat{b}_{MQO}$, e essa magnitude sempre seria zero (teste teria poder trivial).
	\end{itemize}
	\item Não é difícil ver que a magnitude da covariância entre o resíduo do estimador de 2SLS e $X$ depende crucialmente da diferença:
	$$\hat{b}_{MQO} -\hat{b}_{2SLS}\, .$$
	\vspace{-1em}
	\begin{itemize}
		\item Aplicação do {\color{blue}princípio de Hausman} para a construção de testes, em que comparamos dois estimadores, sendo que um deles é consistente sob a nula e inconsistente sob a alternativa (MQO), enquanto o outro é consistente sob a nula e a alternativa (2SLS) (ver Wooldridge).
	\end{itemize}
\end{itemize}
\end{frame}

\begin{frame}{Teste de restrições sobreidentificadoras}
\begin{itemize}
	\item Quando $l > k$, é possível construir um teste para a hipótese nula:
	$$\mathbb{E}[Zu]=0$$
	contra a alternativa de que $$\mathbb{E}[Zu]\neq0$$ com base em uma estatística de Wald que meça a magnitude de:
	
		$$\frac{1}{n}\sum_{i=1}^n Z_i (Y_i - X_i'\hat{b}_{2SLS})$$
		\begin{itemize}
			\item Se $l = k$, essa magnitude é igual a zero, e poder do teste seria trivial.
		\end{itemize}
		\item Teste baseado no fato de que há mais instrumentos que o necessário para a identificação (restrições sobreidentificadoras).
		\begin{itemize}
			\item Rejeição da nula não aponta \textbf{qual} instrumento é inválido.
			\item Ver Wooldridge para a construção do teste.
		\end{itemize}
\end{itemize}
\end{frame}

\begin{frame}{interpretação sob heterogeneidade}
	\begin{itemize}
		\item Os resultados de identificação apresentados anteriormente suponham um modelo causal linear.
		\item \citet{Imbens1994} estudam a interpretação do estimando de Wald sob um modelo causal heterogêneo.
		\item Vamos considerar um resultado de interesse $Y$, causa binária $D \in \{0,1\}$ cujo efeito queremos aferir, e um instrumento binário $Z \in \{0,1\}$.
		\item Definimos os seguintes resultados potenciais, associados a manipulações hipotéticas das causas e dos instrumentos:
		
		$$Y(d,z), \quad (d,z) \in \{0,1\}^2$$
		$$D(z), \quad z \in \{0,1\}^2$$
		\item As variáveis observáveis são dadas por:
		{\small 
		$$D = Z D(1)+(1-Z) D(0)\, ,$$
		$$Y = ZD Y(1,1) + (1-Z)DY(1,0) + Z(1-D) Y(0,1) + (1-Z)(1-D) Y(0,0)$$
	}
	\end{itemize}
\end{frame}
\begin{frame}{Hipóteses sobre o modelo causal heterogêneo}
\begin{itemize}
	\item 	\citet{Imbens1994} supõem as seguintes hipóteses sobre o modelo causal:
	\begin{enumerate}
		\item \textbf{Exclusão:} $$Y(d,0)= Y(d,1) =: Y(d)\, ,\quad d \in \{),1\}$$.
		\item \textbf{Independênca} $Z$ é independente de $\{Y(0), Y(1), D(1), D(0)\}$.
		\item \textbf{Relevância}: $$\frac{\operatorname{cov}(D,Z)}{\operatorname{var}(Z)} = \mathbb{E}[D|Z=1] - \mathbb{E}[D|Z=0]\neq 0$$.
		\item  \text{Monotonicidade:} ou $\mathbb{P}[D(1)\geq D(0)]=1$, ou $\mathbb{P}[D(1)\leq D(0)]=1$.
	\end{enumerate}
	\begin{itemize}
		\item Hipótese de monotonicidade requer que tratamento exerça um efeito unidirecional sobre o tratamento $D$ na população de interesse.
		\begin{itemize}
			\item Ou manipulação de $Z=0$ para $Z=1$ aumenta (fracamente) o valor de $D$ para todos os indivíduos na população, ou diminui.
		\end{itemize}
	\end{itemize}
\end{itemize}
\end{frame}

\begin{frame}{\citet{Imbens1994}}
	\begin{proposition}
		Sob as hipótese 1-4, estimando de Wald:
		
		$$\gamma_{\text{Wald}} =\frac{\operatorname{cov}(Y,Z)}{\operatorname{cov}(D,Z)} =\frac{\mathbb{E}[Y|Z=1] -\mathbb{E}[Y|Z=0]}{\mathbb{E}[D|Z=1]-\mathbb{E}[D|Z=0]}$$
		é igual ao efeito médio do tratamento na subpopulação induzida ao tratamento pelo instrumento (\textit{compliers}), isto é:
		
		$$\gamma_{\text{Wald}} =  \frac{\mathbb{E}[(Y(1)-Y(0))\mathbf{1}_{D(1)\neq D(0)}]}{\mathbb{P}[D(1)\neq D(0)]} =\mathbb{E}[Y(1)-Y(0)|D(1)\neq D(0)]$$
	\end{proposition}
	\begin{itemize}
		\item Estimando identifica um efeito médio de tratamento local (LATE).
		\item Existem extensões para instrumentos discretos \citep{Imbens1994}, tratamentos ordenados \citep{Angrist1995}, tratamentos contínuos \citet{Angrist2000}, especificações com controles (2SLS) \citep{Stoczynski2022,blandhol2022tsls}, instrumentos contínuos \citep{Alvarez2024}.
	\end{itemize}
\end{frame}
\appendix
\begin{frame}[allowframebreaks]
	\printbibliography
\end{frame}

\end{document}





% !TeX document-id = {0be8c18c-9430-4e9a-bdd9-12beadebfebc}
% !TeX TXS-program:bibliography = txs:///biber
\documentclass[11pt]{beamer}

\usepackage[brazilian]{babel}

\uselanguage{portuguese}
\languagepath{portuguese}
\deftranslation[to=portuguese]{Theorem}{Teorema}
\deftranslation[to=portuguese]{theorem}{teorema}
\deftranslation[to=portuguese]{Example}{Exemplo}
\deftranslation[to=portuguese]{example}{exemplo}
\deftranslation[to=portuguese]{Lemma}{Lema}
\deftranslation[to=portuguese]{lemma}{Lema}
\deftranslation[to=portuguese]{Corollary}{Corolário}
\deftranslation[to=portuguese]{corollary}{corolário}
%\deftranslation[to=portuguese]{and}{e}
\newtheorem{proposition}{Proposição}
\newtheorem{assumption}{Hipótese}

\usepackage[utf8]{inputenc}
\usepackage[T1]{fontenc}
\usepackage{lmodern}
\usepackage{amsmath}
\usepackage{amssymb}
\usepackage{mathtools}
\usepackage{color}
\usepackage{pgfplots}
\usepackage{tikz}
\usepackage{subcaption}
%\usepackage{appendixnumberbeamer}

\newenvironment{transitionframe}{
	\setbeamercolor{background canvas}{bg=yellow}
	\begin{frame}}{
	\end{frame}
}
\usetheme{default}
\usefonttheme{structuresmallcapsserif}

%% I use a beige off white for my background
\definecolor{MyBackground}{RGB}{255,253,218}
\useinnertheme[shadow]{rounded}
\setbeamercolor{block title}{bg=MyBackground}
\setbeamercolor{block body}{bg=MyBackground}
\setbeamercolor{example title}{bg=MyBackground}
\setbeamercolor{example body}{bg=MyBackground}


\newcommand{\blue}[1]{\textcolor{blue}{#1}}
\newcommand{\red}[1]{\textcolor{red}{#1}}
\newcommand{\purple}[1]{\textcolor{purple}{#1}}
\newcommand{\gray}[1]{\textcolor{gray}{#1}}
\setbeamertemplate{navigation symbols}{}
%\setbeamertemplate{page number in head/foot}[appendixframenumber]

%\usepackage{graphics}
\usepackage{graphicx}

\definecolor{blue_emph}{RGB}{0,114,178}
\definecolor{red}{RGB}{213,94,0}
\definecolor{yellow}{RGB}{240,228,66}
\definecolor{green}{RGB}{0,158,115}
\definecolor{purple}{RGB}{204,121,167}
\definecolor{orange}{RGB}{230,159,0}
\definecolor{lightblue}{RGB}{86,180,233}

%\setbeamercolor{frametitle}{fg=blue}
%\setbeamercolor{title}{fg=blue}
\setbeamertemplate{footline}[frame number]
\setbeamertemplate{navigation symbols}{} 
\setbeamertemplate{itemize items}{-}
%\setbeamercolor{itemize item}{fg=blue}
%\setbeamercolor{itemize subitem}{fg=blue}
\setbeamertemplate{enumerate items}[default]
%\setbeamercolor{enumerate subitem}{fg=blue}
\setbeamercolor{button}{bg=MyBackground,fg=blue}
\usefonttheme{structuresmallcapsserif}

%\setbeamercolor{section in toc}{fg=blue}
%\setbeamercolor{subsection in toc}{fg=red}
\setbeamersize{text margin left=1em,text margin right=1em} 


\usepackage{appendixnumberbeamer}
\usepackage{pdfpages}
\usepackage[
backend=biber,
uniquename=false,
uniquelist=false,
style=authoryear,
natbib=true
]{biblatex}
\addbibresource{../bibliography.bib}

\newenvironment{wideitemize}{\itemize\addtolength{\itemsep}{10pt}}{\enditemize}
\newenvironment{wideenumerate}{\enumerate\addtolength{\itemsep}{10pt}}{\endenumerate}
\newenvironment{halfwideitemize}{\itemize\addtolength{\itemsep}{0.5em}}{\enditemize}
\newenvironment{halfwideenumerate}{\enumerate\addtolength{\itemsep}{0.5em}}{\endenumerate}


\author{Luis A. F. Alvarez}
\title{Econometria I}
\subtitle{Variáveis instrumentais}
%\logo{}
%\institute{}
\date{\today}
%\subject{}
%\setbeamercovered{transparent}

\def\signed #1{{\leavevmode\unskip\nobreak\hfil\penalty50\hskip2em
		\hbox{}\nobreak\hfil(#1)%
		\parfillskip=0pt \finalhyphendemerits=0 \endgraf}}

\newsavebox\mybox
\newenvironment{aquote}[1]
{\savebox\mybox{#1}\begin{quote}}
	{\signed{\usebox\mybox}\end{quote}}

\begin{document}

\begin{frame}[plain]
	\maketitle
\end{frame}
\begin{frame}{Instrumento}
	\begin{itemize}
		\item Considere um modelo linear {\color{blue}causal} da forma:
		
		$$Y = X\beta + U\, ,$$
		onde $X$ é uma causa observada escalar, $U$ são causas não observadas, e $\beta$ é o efeito causal de $X$.
		\item Considere uma situação em que não é razoável supor que $\operatorname{cov}(X,U) = 0$.
		\begin{itemize}
			\item Nesse caso, a inclinação de $X$ no melhor preditor linear de $Y$ em $X$ e um intercepto não identificará $\beta$, de modo que o estimador de MQO de $Y$ em $X$ e um intercepto não estimará consistentemente o efeito causal de interesse.  
		\end{itemize}
		\item Suponha que observemos uma variável $Z$ que satisfaz:
		\begin{itemize}
			\item (Relevância) $\operatorname{cov}(Z,X)\neq0$.
			\item (Exogeneidade ou exclusão)  $\operatorname{cov}(X,U)=0$
		\end{itemize}
		\item À variável $Z$ damos o nome de {\color{green}instrumento}:
		\begin{itemize}
			\item Trata-se de variável que exibe associação, na população, com $X$ (relevância), e cuja \textbf{única associação com $Y$ se dá através de $X$} (exogeneidade ou exclusão).
		\end{itemize}
\end{itemize}
	\end{frame}
\begin{frame}{Estimando de Wald e Identificação sob variável instrumental}
\begin{itemize}
	\item Defina  \textbf{o estimando de Wald} como o \textbf{parâmetro}:
	
	$$\gamma_{\text{Wald}}  \coloneqq \frac{\operatorname{cov}(Z,Y)}{\operatorname{cov}(Z,X)}\,,$$
	\item Sob as hipóteses de relevância e exogeneidade do instrumento, estimando identifica $\beta$, i.e.:
	
	$$\gamma_{\text{Wald}} = \beta\, .$$
\end{itemize}
\end{frame}
\begin{frame}{Estimação e Inferência}
	\begin{itemize}
		\item Dada uma amostra aleatória $(Y_i,X_i,Z_i)\sim (Y,X,Z)$, resultado anterior sugere que estimemos o efeito causal como:
		
		$$\hat{\gamma}_{\text{Wald}} = \frac{\widehat{\operatorname{cov}(Z,Y)}}{\widehat{\operatorname{cov}(Z,X)}} = \frac{\hat{b}_{Y,Z}}{\hat{b}_{Z,X}}\, ,$$
		onde $\hat{b}_{S,Z}$ é o estimador de MQO para o coeficiente de $S$ numa regressão de $S$ num intercepto e $Z$.
		\item Sob condições de regularidade da aula anterior, vimos que, com $n \to \infty$ $$\hat{b}_{Y,Z} \overset{p}{\to} \frac{\operatorname{cov}(Y,Z)}{\mathbb{V}(Z)}$$
		
		$$\hat{b}_{X,Z} \overset{p}{\to} \frac{\operatorname{cov}(X,Z)}{\mathbb{V}(Z)}$$
		de onde segue, por aplicação do teorema do mapa contínuo que, sob as condições de relevância e exogeneidade:
		\vspace{-0.2em}
		$$\hat{\gamma}_{\text{Wald}}  \overset{p}{\to} \gamma_{\text{Wald}} = \beta\, .$$
		
	\end{itemize}
\end{frame}
\begin{frame}{Testando relevância}
	\begin{itemize}
		\item Note que, como estimador de $\hat{b}_{X,Z}$ é consistente para $\frac{\operatorname{cov}(X,Z)}{\mathbb{V}(Z)}$, podemos utilizar esse estimador para realizar um teste da nula de que que o {instrumento} \textbf{não} é relevante.
		\item Por outro lado, em nosso ambiente com uma variável instrumental e um modelo causal linear, a hipótese de {exclusão} é \textbf{intestável}.
		\begin{itemize}
			\item Devemos suplementar a análise empírica com uma argumentação de porquê o único mecanismo através do qual variações em $Z$ produzem variações em $Y$ é através de $X$.
		\end{itemize}
	\end{itemize}
\end{frame}

\begin{frame}{Caso geral}
\begin{itemize}
	\item Vamos agora considerar um modelo causal geral da forma:
$$Y = X'\beta + U$$
onde $X$ é um vetor de $k$ causas observadas.
\item Vamos supor a existência de um vetor $Z$  de $l$ instrumentos que satisfaz as hipóteses:
\begin{assumption}[H1-Relevância]
	$\mathbb{E}[ZX']$ tem posto $k$.
\end{assumption}

\begin{assumption}[H2-Exogeneidade]
	$\mathbb{E}[ZU]=0$
\end{assumption}
\end{itemize}

\begin{itemize}
	\item Note que a hipótese de exogeneidade é equivalente a $\operatorname{cov}(Z,\epsilon)=0$ quando $X$ inclui um intercepto, pois nesse caso podemos supor que $U$ tem média zero ``absorvendo sua média'' ao intercepto.
\end{itemize}

\end{frame}

\begin{frame}{Variáveis Endógenas e Exógenas}
\begin{itemize}
	\item A condição de relevância implica que necessitamos de $l\geq k$ variáveis que exibam suficiente variação com as causas $X$, e que não exibam associação com as causas $U$.
	\begin{itemize}
		\item Se uma entrada $X_j$ é {\color{blue}exógena} por hipótese, i.e. $\mathbb{E}[X_j U] = 0$, podemos incluí-la entre os instrumentos $Z$.
		\item Por outro lado, para cada variável $X_s$  {\color{blue}endógena}, i.e. tal que potencialmente $\mathbb{E}[X_j U] \neq 0$, precisamos de pelo menos um instrumento que exiba associação com as causas observadas, mas não com as causas não observadas.
	\end{itemize}
\end{itemize}
\end{frame}

\begin{frame}{Identificação dos efeitos no modelo linear geral}
\begin{itemize}
	\item Sob as hipóteses H1 e H2, {\color{red}para qualquer matriz $A$ $k\times l$} de posto cheio, temos que:
	$$ \gamma_A := (\mathbb{E}[AZX'])^{-1}\mathbb{E}[AZY] = \beta$$
	\item Note que, o resultado anterior sugere que, para uma amostra aleatória $(Y_i,X_i, Z_i)\sim (Y,X,Z)$ e uma dada matriz $A$, podemos estimar $\beta$ por:
	
	$$\hat{\gamma}_A = \left(\sum_{i=1}^n \hat{A} Z_i X_i'\right)^{-1}\sum_{i=1}^n \hat{A} Z_i Y_i\, = (\hat{A} \boldsymbol{Z}'\boldsymbol{X})^{-1}\hat{A} \boldsymbol{Z}' \boldsymbol{y}\, ,$$
	onde $\hat{A}$ é um estimador da matriz $A$ e:
	$$\boldsymbol{Z} = \begin{bmatrix}
		Z_1' \\
		Z_2' \\
		\vdots \\
		Z_n'
	\end{bmatrix}\, , \quad \boldsymbol{X} = \begin{bmatrix}
	X_1' \\
	X_2' \\
	\vdots \\
	X_n'
	\end{bmatrix} $$
\end{itemize}

\end{frame}
\end{document}




